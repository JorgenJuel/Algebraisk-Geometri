\documentclass{article}
\usepackage[utf8]{inputenc}

\usepackage{amsthm}
\usepackage{amsmath}
\usepackage{enumitem}

\newtheorem{definisjon}{Definisjon}[section]
\newtheorem{teorem}{Teorem}[section]

\title{Algebraisk Geometri}
\author{Jørgen }
\date{January 2019}

\begin{document}


\maketitle

\section{Grunnleggende algebra}


\begin{definisjon}
En \textbf{gruppe} (G, +) er en mengde $G$ sammen med en binæroperasjon $+$, slik at:
\begin{itemize}[noitemsep]
    \item $(a+b) + c = a + (b + c)$
    \item $ \exists e \in G: a+e = e = e+a,  \forall a \in G $
    \item $\forall a \in G: \exists b \in G: a + b = e$
\end{itemize}
Dersom elementene i tillegg er kommutative, $a +b = b + a$ sier vi at gruppen er \textbf{abelsk}.
\end{definisjon}

\begin{definisjon}
En \textbf{ring} (R, +, $\times$) er en mengde R sammen med to binæroperasjoner +, $\times$, slik at følgende gjelder:
\begin{itemize}[noitemsep]
    \item (R, +) er en abelsk gruppe
    \item Assosiativitet over $\times$: $(ab)c = a(bc)$
    \item Distributivitet over $\times$: $a(b + c) = ab + ac$ og $(a+b)c = ac + bc$
\end{itemize}
Dersom ringen $R$ er kommutativ med egenskapen $ab=0 \Rightarrow a = 0 \text{ eller } b = 0$ så kaller vi den et \textbf{integritetsområde}.
\end{definisjon}

\begin{definisjon}
En ikke-tom delmengde $S$ av en ring $R$ kalles et \textbf{ideal} av $R$ dersom 
\begin{enumerate}[noitemsep]
    \item $a,b \in S \Rightarrow a-b \in S$
    \item $a \in S, r \in R \Rightarrow ar \in S$ and $ra \in S$
\end{enumerate}
\end{definisjon}

\begin{definisjon}
Et ideal $P$ i en ring $R$ kalles et primideal dersom den har følgende egenskap: Hvis $A$ og $B$ er idealer i $R$ slik at $AB \subseteq P$ så er enten $A \subseteq P$ eller $B \subseteq P$. \\
Dersom $R$ er kommutativ vil et ideal være et primideal hvis og bare hvis $$
ab \in P, a,b \in R \Rightarrow a \in P \text{ eller } b \in P
$$
\end{definisjon}

Vi skal nå se nærmere på hvordan vi kan utvide et integritetsområde til en kropp.
La oss først definere en ekvivalensrelasjon $\sim$ over $R \times R \setminus \{0\}$ der $(a,b) \sim (a', b')$ dersom det finnes $c'' \in R \setminus \{0\}$ slik at $c''(rb' - r's) = 0$. Med ekvivalensrelasjonen kan vi la $a/b$ definere ekvivalensklassen som bestemmes av $(a,b)$, la så $R_S$ betegne mengden av alle ekvivalensklasser. \\

 Vi definerer så aritmetikken på følgende måte: $a/b + c/d = (ad + cb)/(bd)$ og $a/b \cdot c/d = (ac)/(bd)$. $R_S$ med denne aritmetikken danner en Ring med enhet. 
 
 \begin{teorem}
 Ethvert kommutativt integritetsområde $R$ kan inkluderes i en kropp $R_s$ (som kalles kroppen av brøker). 
 \textbf{Bevis:} Side 226, nøstes opp til 225.
 \end{teorem}
\section{Algebraisk Geometri}
\begin{definisjon}
Et affine n-rom $\mathcal{A}^n(k)$ er alle $n$-tuppler i kroppen $k$. 
\begin{equation*}
    \mathcal{A}^n(k) = \{P = (a_1, \ldots, a_n) \in k^n \}
\end{equation*}
Ofte forkortes dette til bare $\mathcal{A}^n$ dersom det ikke er noe tvil om hvilken kropp det er snakk om. Elementene $P \in \mathcal{A}^n$ kalles punkter.
\end{definisjon}
\begin{definisjon}En affine algebraisk mengde $V(S)$ er en delmengde av $\mathcal{A}^n$ som er nullpunkter til funksjonene i mengden $S$. 
\begin{equation*}
 V(S) = \{P \in \mathcal{A}^n | F(P) = 0, \forall F \in S\}   
\end{equation*}
\end{definisjon}
Det er ikke uvanlig at $S$ bare inneholder én funksjon. Hvis det er tilfellet vil den algebraiske mengden kun være nullpunktene til denne funksjonen. \textbf{Merk:} Nullpunktene til den tomme mengden $V(\emptyset)$ er hele det affine n-rommet. 
\begin{definisjon}
Idealet til en algebraisk mengde er ganske enkelt idealet generert av alle polynomene som tilfredsstiller alle nullpunktene.
\begin{equation*}
    I(V) = {F \in k[X_1, \ldots, X_n] | F(P) = 0, \forall P \in V}
\end{equation*}
\end{definisjon}
\begin{definisjon}En affine algebraisk mengde V kalles en \textit{varietet} dersom idealet $I(V)$ er et primideal i $\overline{K}[X_1, \ldots, X_n]$. Vi skriver ofte $V/K$ for å indikere at $V$ er en varietet definert over $K$.
\end{definisjon}
\textit{Merk:} Det er ikke nok at den er et primideal i $K[X_1, \ldots, X_n]$
\textbf{Moteksempel:} Idealet $(X_1^2 - 2X_2^2)$ er et primideal i $\mathcal(Q)[X_1, X_2]$.
\begin{definisjon}
La $V/K$ være en varietet, da er den affine \textbf{koordinatringen} $K[V]$ til $V/K$ definert på følgende måte
\begin{equation*}
    K[V] = \frac{K[X_1, \ldots, X_n]}{I(V/K)}
\end{equation*}
\end{definisjon}
Siden $V/K$ er en varietet er dens ideal $I(V/K)$ et primideal, og når $K[X_1, \ldots, X_n]$ er en ring vil $K[V]$ være et integritetsområde. Det er derfor naturlig å utvide dette til en kropp.
\begin{definisjon}
Funksjonskroppen $K(V)$ er ganske enkelt kroppsutvidelsen til $K[V]$ slik at alle elementer $F \in K(V)$ kan skrives som $F = G + H$ for $G, H \in K[V]$.
\end{definisjon}
\end{document}
